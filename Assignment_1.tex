\documentclass{article}\usepackage[]{graphicx}\usepackage[]{color}
%% maxwidth is the original width if it is less than linewidth
%% otherwise use linewidth (to make sure the graphics do not exceed the margin)
\makeatletter
\def\maxwidth{ %
  \ifdim\Gin@nat@width>\linewidth
    \linewidth
  \else
    \Gin@nat@width
  \fi
}
\makeatother

\definecolor{fgcolor}{rgb}{0.345, 0.345, 0.345}
\newcommand{\hlnum}[1]{\textcolor[rgb]{0.686,0.059,0.569}{#1}}%
\newcommand{\hlstr}[1]{\textcolor[rgb]{0.192,0.494,0.8}{#1}}%
\newcommand{\hlcom}[1]{\textcolor[rgb]{0.678,0.584,0.686}{\textit{#1}}}%
\newcommand{\hlopt}[1]{\textcolor[rgb]{0,0,0}{#1}}%
\newcommand{\hlstd}[1]{\textcolor[rgb]{0.345,0.345,0.345}{#1}}%
\newcommand{\hlkwa}[1]{\textcolor[rgb]{0.161,0.373,0.58}{\textbf{#1}}}%
\newcommand{\hlkwb}[1]{\textcolor[rgb]{0.69,0.353,0.396}{#1}}%
\newcommand{\hlkwc}[1]{\textcolor[rgb]{0.333,0.667,0.333}{#1}}%
\newcommand{\hlkwd}[1]{\textcolor[rgb]{0.737,0.353,0.396}{\textbf{#1}}}%

\usepackage{framed}
\makeatletter
\newenvironment{kframe}{%
 \def\at@end@of@kframe{}%
 \ifinner\ifhmode%
  \def\at@end@of@kframe{\end{minipage}}%
  \begin{minipage}{\columnwidth}%
 \fi\fi%
 \def\FrameCommand##1{\hskip\@totalleftmargin \hskip-\fboxsep
 \colorbox{shadecolor}{##1}\hskip-\fboxsep
     % There is no \\@totalrightmargin, so:
     \hskip-\linewidth \hskip-\@totalleftmargin \hskip\columnwidth}%
 \MakeFramed {\advance\hsize-\width
   \@totalleftmargin\z@ \linewidth\hsize
   \@setminipage}}%
 {\par\unskip\endMakeFramed%
 \at@end@of@kframe}
\makeatother

\definecolor{shadecolor}{rgb}{.97, .97, .97}
\definecolor{messagecolor}{rgb}{0, 0, 0}
\definecolor{warningcolor}{rgb}{1, 0, 1}
\definecolor{errorcolor}{rgb}{1, 0, 0}
\newenvironment{knitrout}{}{} % an empty environment to be redefined in TeX

\usepackage{alltt}

\title{Week 1 assessment}

\renewcommand\labelenumi{\thesection.\arabic{enumi}\hfil}



\IfFileExists{upquote.sty}{\usepackage{upquote}}{}
\begin{document}
\maketitle
\section{Getting help}


\begin{enumerate}
\item What would you do if you wanted to find out what a function does?
\item What would you do if you wanted to work out how to do something in R?
\item What are the best practices for getting help from someone?
\end{enumerate}

\section{Assignment}
\begin{enumerate}
\item Assign a vector to the object \texttt{x}. 
\begin{enumerate}
\item What storage mode is it? (Place this as a comment)
\item Assign \texttt{l} to be \texttt{x}'s length.
\item Assign \texttt{x.1a} to be \texttt{x}'s first element. 
\item Assign \texttt{x.1b} to be \texttt{x}'s first element as well, but do it in a different way.
\item Write a line of code that verifies \texttt{x.1a} and \texttt{x.1b} are equal.
\item Assign \texttt{x.2} to be every second element of \texttt{x}.
\item Assign \texttt{x.n} to be every element of \texttt{x} that has an unambiguous numeric representation. Comment what you interpret ``unambiguous numeric representation to mean''.
\item Concisely express how many elements of \texttt{x} are whole numbers. (Look at \texttt{?is.integer}, especially the end of the help file).
\end{enumerate}
\item What is the difference between \verb"<-" and \verb"=" in assignment?
\end{enumerate}

\section{IO}
\begin{enumerate}
\item 
\end{enumerate}



\end{document}
